\documentclass[10pt, a4paper]{article}
\usepackage{amsmath, amsthm, amssymb}
\usepackage[utf8]{inputenc}
\usepackage{hyperref}

\newtheorem{theorem}{Theorem}[section]
\newtheorem{lemma}[theorem]{Lemma}

\title{Emergent Machine Pedagogy: A Formal Framework}
\author{[Your Name]}
\date{\today}

\begin{document}
\maketitle

\begin{abstract}
    This document outlines the formal theoretical foundations of Emergent Machine Pedagogy (EMP), a framework for building inventive AI systems capable of surpassing imitation learning.
\end{abstract}

\section{The Core Theorems of EMP}

\begin{theorem}[The Imitation Efficacy Ceiling]
    Let \(\mathcal{P}_{expert}\) be a finite set of expert policies. Let \(\text{conv}(\mathcal{P}_{expert})\) be the convex hull of these policies. For any policy \(\pi \in \text{conv}(\mathcal{P}_{expert})\), its expected efficacy \(E[\pi]\) is bounded by the efficacy of the single best expert in the set:
    \[ E[\pi] \leq \max_{\pi_i \in \mathcal{P}_{expert}} E[\pi_i] \]
\end{theorem}

\begin{theorem}[The Discovery-Efficacy Tradeoff]
    % Statement to be added
\end{theorem}

\begin{theorem}[The Critical Diversity Threshold]
    % Statement to be added
\end{theorem}

\end{document}